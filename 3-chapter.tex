\newpage
\section{\Large Классификация температурных данных компьютерного моделирования}\vspace{-7mm}
\subsection{Данные для классификации}
В работе использовались данные компьютерного моделирования яркостной температуры молочных желез больных и здоровых пациентов. Данные были представлены в виде девяти значений температуры на поверхности кожи и девяти значений внутренней температуры, согласно методике обследования методом радиотермометрии~\cite{fear2000}~\cite{bardati}. Схема расположения точек при замере температур представлена на рисунке~\ref{ris:rtm-res-scheme}~\cite{vesninSovMicrowave}. Отдельным атрибутом является класс модели. Для здоровых моделей значения класса было равно нулю, а для больных -- единице. Исходя из количества классов, классификацию в данной работе можно считать бинарной.
\\
\imgh{0.75\linewidth}{rtm-res-scheme}{Пример данных температурных данных компьютерного моделирования, где в столбце «target» здоровые — «0», больные — «1»}
\par
Для исследования были взяты температурные данные моделей с радиусом опухоли 0.5 см и 0.75 см. Данные были представлены в виде двух CSV-файлов, в которых находилось по 160 моделей для каждого размера опухоли соответственно (рисунок~\ref{ris:data_example}). Одна половина моделей состояла из здоровых пациентов, а другая из больных. На подготовительном этапе данные были разбиты на обучающую и тестовую выборки в соотношении один к четырем.
\\
\imgh{1\linewidth}{data_example}{Пример данных температурных данных компьютерного моделирования, где в столбце «target» здоровые — «0», больные — «1»}

%------------------------------------------------------------------------------------------------------

\subsection{Частотный анализ классифицируемых данных}
Перед классификацией температурных данных был проведен частотный анализ для исследования распределения температуры в зависимости от того, в какой точке она была замерена и есть ли у данной модели опухоль. Частотный анализ был произведен с помощью библиотеки Pandas, а графическое отображение результатов с помощью Matplotlib. Точки, которые были проанализрованы соответствуют точкам на схеме измерений согласно методу РТМ (рисунок~\ref{ris:rtm-res-scheme}). Код программы для отображения результатов частотного анализа находится в приложении (листинг А.2).
\par
Сначала для исследования была взяла точка 0ртм. Результаты исследования представлены в виде диаграммы частот (рисунок~\ref{ris:0rtm}), где верхняя диаграмма -- это здоровые пациенты, а нижняя -- больные.
\imgh{1\linewidth}{0rtm}{Диаграмма частот температур в точке 0ртм. Верхняя диаграмма — здоровые пациенты, нижняя — больные}
\par
Изучив диаграммы, можно заметить, что в данной точке у пациентов чаще встречаются более высокие температуры, нежели у здоровых. Точка 0ртм больше всего подверждена воздействию от опухоли в температурном смысле, так как находится ровно в центре молочной железы.
\par
Следующей для исследования была рассмотрена точка 3ртм, диаграмма частот температур которой представлена на рисунке~\ref{ris:3rtm}. Для данной точки можно наблюдать схожую картину с точкой 0ртм -- тут так же больше значений со средней температурой у больных пациентов, несмотря на то, что эта точка не является центральной.
\imgh{1\linewidth}{3rtm}{Диаграмма частот температур в точке 3ртм. Верхняя диаграмма — здоровые пациенты, нижняя — больные}
\par
Также была рассмотрена точка 7ртм, для которой как видно из диаграммы частот температур (рисунок~\ref{ris:7rtm}) ситуация аналогична с точками 0ртм и 3ртм.
\imgh{1\linewidth}{7rtm}{Диаграмма частот температур в точке 7ртм. Верхняя диаграмма — здоровые пациенты, нижняя — больные}

%------------------------------------------------------------------------------------------------------

\subsection{Результаты классификации моделей с опухолью радиусом 0.5 см}
Для моделей с разными размерами опухоли отдельно была посчитана точность определения класса и построены круговые диаграммы для наглядности, где как «1» отмечены верно классифицированные модели, а как «0» -- неверно классифицированные.
\par
Как видно из диаграммы на рисунке~\ref{ris:05}, для моделей с радиусом опухоли 0.5 см лучший результат показал метод SVM, точность определения классов которого равна 57.5\%. Хуже всего для данных моделей показал себя наивный байесовский классификатор с точностью классификации 27.5\%
\imgh{1\linewidth}{05}{Диаграммы с точностью определения класса для трех методов классификации моделей с размером опухоли R=0.5 см («1» – классифицировано верно, «0» – классифицировано неверно)}

%------------------------------------------------------------------------------------------------------
\subsection{Результаты классификации моделей с опухолью радиусом 0.75 см}
Рассмотрим результаты классификации для моделей с радиусом опухоли 0.75 см (рисунок~\ref{ris:075}). Как видно из диаграмм, для данных моделей наивный байесовский классификатор сработал лучше, для для моделей с радиусом опухоли 0.5 см, показав результат точности равный 70\%. Точность определения классов методом SVM опять оказалась лучше, чем у остальных -- 72.5%.
\par
Исходя из этого, можно сделать вывод, что точность классификации сильно зависит от размеров опухоли и выбранного алгоритма классификации.
\imgh{1\linewidth}{075}{Диаграммы с точностью определения класса для трех методов классификации моделей с размером опухоли R=0.75 см («1» – классифицировано верно, «0» – классифицировано неверно)}

%------------------------------------------------------------------------------------------------------

\subsection{Результаты классификации для всех моделей вместе}
В качестве заключительного эксперимента было принято решение смешать вместе данные моделей с радиусом 0.5 см и 0.75 см. В качестве обучающей выборки использовались 180 моделей, а в качестве тестовой -- 60 моделей. После чтения из файла данные были перемешаны между собой. 
\par
Результаты представлены на диаграмме (рисунок~\ref{ris:075}). В этот раз лучшим по точности определения класса стал наивный байесовский классификатор с точностью 70\%, которая не изменилась по сравнению с предыдущим экспериментом. А метод  SVM, показавший лучший результат в предыдущих двух экспериментах, в этот раз определил классы моделей с точностью 66.7\%, такой же как и метод k-ближайших соседей.
\imgh{1\linewidth}{all}{Диаграммы с точностью определения класса для трех методов классификации моделей с размером опухоли R=0.5 см и R=0.75 см вместе («1» – классифицировано верно, «0» – классифицировано неверно)}

%------------------------------------------------------------------------------------------------------
%------------------------------------------------------------------------------------------------------
